% Hyperref is needed to create links between usage of an entry and its
% descriptive table. If you don't want links, just comment the line.
\usepackage{hyperref}

% If you want the glossary to be hidden from the table of contents:
%\usepackage{glossaries}
% If you want the glossary to be visible in the table of contents:
%\usepackage[toc]{glossaries}

\usepackage[toc,nopostdot,nonumberlist]{glossaries}

\setacronymstyle{long-short}

\makeglossaries

% To declare a new entry in the document body:
%   \newacronym{TOTO}{TOTO}{An acronym.}
%
% Make use of an entry in the document:
%   \gls{TOTO}
%
% To print the glossary at the right place in the document:
%    \printglossaries


% ------------------------------------------------------------------------------
% To have a fancy glossary
%   See: https://tex.stackexchange.com/a/394983
%
\usepackage{tipa}
\usepackage{array}
\usepackage{longtable}

\newglossarystyle{glossaryStyle_custom}
{%
    % longtable with three columns:
    \renewenvironment{theglossary}%
    {
        \begin{longtable}{|l|l|>{\raggedright}p{.5\linewidth}|}
    }%
    {
        \end{longtable}
    }%

    % Header content
    \renewcommand*{\glossaryheader}
    {
        \hline
        \textbf{Acronyme} &
        \textbf{Correspondance} &
        \textbf{Description}\tabularnewline
        \hline
    }%

    % Groups heading content
    \renewcommand*{\glsgroupheading}[1]
    {
        % no group headings:
    }%


    % main (level 0) entries displayed in a row
    \renewcommand{\glossentry}[2]
    {
        % name in bold:
        \glsentryitem{##1}\glstarget{##1}{\textbf{\glossentryname{##1}}} &

        % symbol in square brackets
        %[\glossentrysymbol{##1}] &
        \glsentrytext{##1} &

        % description in italic
        \textit{\glossentrydesc{##1}}\tabularnewline

        % Draw an horizontal line in the table
        \hline
    }


    % sub-entries (same as main)
    \renewcommand{\subglossentry}[3]
    {
        \glossentry{##2}{##3}
    }


    % blank row between groups if nogroupskip=false
    \ifglsnogroupskip
        \renewcommand*{\glsgroupskip}{}%
    \else
        \renewcommand*{\glsgroupskip}{ & & \tabularnewline}%
    \fi
}

\setglossarystyle{glossaryStyle_custom}



%
% See the thread at
% https://latex.org/forum/viewtopic.php?t=2976
% to create two separate tables for glossary and acronym entries, and link
% an acronym to its corresponding glossary entry.
