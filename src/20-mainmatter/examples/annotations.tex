\chapter{Annotations}

\demo{
    % Name
    footnotes
}
{
    % Description
    To add footnotes.
}
{
    % Usage
    <text>\footnote{<content of footnote>}
}
{
    % Example
    Here is an example text\footnote {Here is the corresponding footnote}.
}



\demo{
    % Name
    ToDo - "Fixme"
}
{
    % Description
    To mark some text as "to be fixed" by highlighting it and add it to the
    "ToDo" list.
}
{
    % Usage
    \fixme{<text to highlight>}{<Comment>}
}
{
    % Example
    This is to show we can \fixme{highlight}{A comment is added (example).}
    something to fix.
}



\demo{
    % Name
    ToDo (simple usage)
}
{
    % Description
    Adds "Post-It"-like comments. It can also generate a list of "todo" with
    references to find them more easily in the document.
}
{
    % Usage
    <text>\todo{<content of footnote>}
}
{
    % Example
    Here is an example text.\todo{Here is the corresponding ToDo}
}



\demo{
    % Name
    ToDo list
}
{
    % Description
    Generates a list of "todo" comments with references to find them more easily
    in the document.
    \paragraph{Note:} It seems to need two builds to be generated correctly...
}
{
    % Usage
    \listoftodos
}
{
    % Example
    (see on next page)
    \listoftodos
}



\demo{
    % Name
    Word-like comments based on "todo"
}
{
    % Description
    Generates comments that look like the ones from Microsoft Word.

    They also are added to the "ToDo list".
}
{
    % Usage
    \comment[<author (optional)>]{<text>}
}
{
    % Example
    My \comment[Joe]{This is a comment.} text.

    And \comment[Joe]{This is another comment.} another one.
}



\section{
    % Name
    Glossaries
}

    \subsection{Description}
    Generates glossaries

    \subsection{Usage}

    % Usage
    Declare new entries (preferably in preable):
    \begin{lstlisting}
    \newglossaryentry{<glossary entry label>}
    {
        name={<glossary entry acronym>},
        text={<glossary entry full name>},
        description={<Descriptive text.>}
    }
    \end{lstlisting}

    or
    \begin{lstlisting}
        \newacronym
        [description={<Descriptive text>.}]
        {<acronym label>}
        {<acronym>}
        {<acronym full text>}
    \end{lstlisting}

    Please note that \texttt{acronym} and \texttt{glossary} entries behave
    differently in the document.



    \paragraph{}Display the glossary:
    \begin{lstlisting}
    \printglossaries
    \end{lstlisting}

    \paragraph{}Make use of an entry in the document:
    \begin{lstlisting}
    \gls{<entry label>}
    \end{lstlisting}


\subsection{Example}
\subsubsection{Input}
In the preamble (inside the \verb|src/05-header_user/glossaries_entries.tex|
file):
\begin{lstlisting}
    \newacronym
    [description={My acronym description.}]
    {acr_label}
    {MyAcronym}
    {MyAcronymText}
\end{lstlisting}


In the document:
\begin{lstlisting}
Acronym usage:\\
Look for the acronym \gls{acr_label} -text and acronym
in parenthesis- in the glossary on next page. On
second usage of the acronym, \gls{acr_label} -only
acronym- should appear. If the \texttt{hyperref}
package is loaded before \texttt{glossaries},
then it will also create links to the glossary entries.


The glossary: (see at the end of the document)
\end{lstlisting}


\subsubsection{Output}

Acronym usage:\\
Look for the acronym \gls{acr_label} -text and acronym
in parenthesis- in the glossary on next page. On
second usage of the acronym, \gls{acr_label} -only
acronym- should appear. If the \texttt{hyperref}
package is loaded before \texttt{glossaries},
then it will also create links to the glossary entries.


The glossary: (see at the end of the document)
